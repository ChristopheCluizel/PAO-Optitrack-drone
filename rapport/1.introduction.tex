\chapter{Introduction}
	Ce PAO s'inscrit dans le semestre 8 du département ASI, c'est-à-dire lors du second semestre d'ASI4. Il est d'une durée d'un semestre et est suivi par M. Guerrero. Le but principal de ce PAO est de prendre en main le nouveau système Optitrack acquis par le département ASI, afin de l'utiliser pour de futurs projets. Nous avons décidé de coupler ce système avec l'ARDrone de la société Parrot. \\

	Ce rapport a pour but de regrouper l'ensemble des informations que nous avons apprises sur le système Optitrack et l'ARDrone, afin qu'une autre équipe puisse reprendre le projet plus aisément par la suite et améliorer ce qui a été réalisé. \\

	Dans ce but, nous présenterons tout d'abord brièvement le système Optitrack et l'ARDrone, ainsi que les objectifs et l'organisation de ce PAO\@. Puis nous réaliserons un tutoriel de prise en main des 2 systèmes et une présentation des réalisations logicielles. Enfin, nous évoquerons les perspectives d'amélioration.
