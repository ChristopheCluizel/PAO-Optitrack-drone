\chapter{Tutoriel}
\section{Prise en main de l'ARDrone}
L'ARDrone est un quadricoptère grand public contrôlable en wifi avec un smartphone. Dans le cadre du PAO, nous avons pour objectif de le faire voler en asservissant sa position dans l'espace grâce au système OptiTrack.
\subsection{Test de vol}
Avant de s'aventurer à commander le drone depuis un PC, nous avons désiré savoir s'il était fonctionnel. Pour cela nous avons téléchargé l'application fournie par Parrot (la marque du drone).\\
\url{https://itunes.apple.com/fr/app/free-flight/id373065271?mt=8}\\
\url{https://play.google.com/store/apps/details?id=com.parrot.freeflight&hl=fr}\\
Une fois la batterie installée à bord, le drone s'allume tout seul et diffuse un réseau wifi dont le SSID ressemble à \textit{ardrone\_XXXX}. Il faut alors connecter le smartphone au réseau wifi du drone et lancer l'application téléchargée.\\
Le drone est simple à piloter, le bouton \textit{take off} fait s'envoler le drone en mode stationnaire à 1 mètre d'altitude. Les sticks présents à l'écran permettent de l'incliner dans la direction souhaitée, de modifier son altitude, et de changer son orientation. Le bouton \textit{land} permet de le faire atterrir.\\
Si le vol d'essai s'est déroulé sans problème, le drone est alors fonctionnel.



\section{Prise en main du système OptiTrack}