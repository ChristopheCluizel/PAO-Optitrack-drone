\chapter{Présentation des systèmes Optitrack et ARDrone}

    % ============= Système Optitrack =============
    \section{Système Optitrack}
        Le système Optitrack est un système permettant de détecter des marqueurs infrarouges grâce à un nombre plus ou moins important de caméras infrarouges. Ce système est composé de caméras (au minimum 3) reliées à un hub usb lui-même relié à un ordinateur et d'un logiciel de traitement des données nommés Motive. \\

        Ce logiciel permet de calibrer les caméras ensembles, de fixer l'origine du repère par rapport aux caméras, de créer des rigidbodies à partir de marqueurs infrarouges et d'obtenir la position des différents marqueurs dans l'espace (et donc des rigidbodies). \\

        Le logiciel Motive peut fonctionner de façon autonome, mais également en tant que serveur couplé à une autre application. En effet, une application cliente peut se connecter à lui afin de récupérer les informations des marqueurs infrarouges et les utiliser de manière extérieure au logiciel Motive. L'architecture suit celle d'une architecture client-serveur.


    % ============= Système ARDrone =============
    \section{Système ARDrone}
        L'ARDrone est un quadricoptère de la marque Parrot. Il est équipé d'un ordinateur ARM 9 cadencé à 468 MHz, d'une RAM de 128 Mo, d'une connectivité Wi-Fi et d'une autonomie de 12 minutes. Il embarque également un accéléromètre 3 axes et un gyroscope 3 axes lui permettant de se stabiliser.\\

        Il est par ailleurs équipé de deux caméras, une frontale et une ventrale aidant à sa stabilisation. Bien qu'il embarque un ordinateur, le drone n'est pas nativement programmable. Il n'est que commandable en orientation, altitude et inclinaison.
