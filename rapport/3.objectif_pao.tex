\chapter{Objectifs du PAO}
	% ============== Objectifs du système Optitrack ==========
	\section{Objectifs du système Optitrack}
		Le système Optitrack est fourni avec une suite de logiciels, Motive, un SDK camera et un SDK NatNet. Les objectifs sont donc les suivants: \\
		\begin{itemize}
			\item Installation des caméras et branchement;
			\item Installation des logiciels et prise en main afin de:
			\begin{itemize}
				\item Calibrer les caméras;
				\item Paramétrer l'environnement 3D, origine, axe;
				\item Reconnaître et positionner un objet dans l'espace 3D\@.
			\end{itemize}
			\item Intégration des calculs de positionnement dans notre propre programme.
		\end{itemize}


	% ============== Objectifs de l'ARDrone =============
	\section{Objectifs de l'ARDrone}
		L'ARDrone est fourni avec une API officielle et un SDK\@. Les objectifs sont donc les suivants: \\
		\begin{itemize}
			\item Installation et prise en main du SDK\@;
			\item Programmation d'instructions de déplacement dans l'espace 3D\@.
		\end{itemize}


	% ============== Objectifs finaux =============
	\section{Objectifs finaux}
		Une fois les objectifs du système Optitrack et de l'ARDrone réalisés séparément, il faut pouvoir les fusionner afin d'obtenir les objectifs finaux suivants: \\
		\begin{itemize}
			\item Détection et positionnement/orientation d'un drone dans l'espace 3D\@;
			\item Programmation de déplacements et trajectoires en fonction des informations envoyées par le système Optitrack.
		\end{itemize}
