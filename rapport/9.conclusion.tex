\chapter{Conclusion}
    Les objectifs finaux de ce PAO étaient de pouvoir détecter un drone, obtenir son positionnement/orientation dans l'espace 3D, afin de pouvoir programmer des déplacements et trajectoires en fonction des informations envoyées par le système Optitrack. \\

    Pour arriver à ces fins, nous avons procédé par étapes successives, à savoir:
    \begin{itemize}
        \item la prise en main du système Optitrack;
        \item la prise en main du drone;
        \item l'intégration des 2 technologies dans une même application;
        \item l'implémentation d'un asservissement à correcteur PD\@;
        \item l'implémentation de suivi d'une trajectoire par points de passage. \\
    \end{itemize}

    Ces différentes étapes ont été découpée en tâches, chacune étant dépendante de la précédente. De plus, chaque étape apportait de nouvelles notions à intégrer; le logiciel Motive d'Optitrack, l'ARDrone, le concept de client-serveur, la notion d'asservissement. \\

    Si le projet est repris, il serait préférable d'installer des filets afin de préserver le drone et l'environnement adjacent. En effet, le drone peut devenir hors de contrôle pour de nombreuses raisons et causer des dommages à lui-même ou à son entourage. \\

    Une salle avec plus d'espace serait intéressante pour permettre des mouvements du drone plus aisés. Cependant, en l'état, la longueur des câbles des caméras Optitrack ne permet pas ce changement. De plus, il faut vérifier certaines contraintes du système comme une luminosité ambiante limitée pour éviter les parasites et la distance maximale entre les caméras et le hub par exemple.

