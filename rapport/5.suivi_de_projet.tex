\chapter{Suivi de projet}
Les comptes-rendus retracent les objectifs, les réalisations, ainsi que le temps estimé pour réaliser ces objectifs.

%============= Compte rendu à la date du 26/11 ==============
% \section{Compte rendu à la date du 26/11}
% Les deux objectifs de ces 2 semaines étaient d'étudier la possibilité d'utiliser la stéréoscopie pour évaluer la distance entre Qbo et les plots et d'avancer la rédaction du rapport.

% %---------------- Alexandre Brehmer -------------------------
% \subsection{Alexandre Brehmer}
% Tâches effectuées:
% \begin{itemize}
% 	\item recherche documentaire sur la stéréo-vision;
% 	\item mise en œuvre de la stéréo-vision sur Qbo;
% 	\item rédaction d'une présentation sur ROS\@;
% 	\item rédaction de l'objectif du PAO\@;
% 	\item rédaction d'un tutoriel sur l'apprentissage du Haar Classifier;
% 	\item rédaction d'un tutoriel sur l'estimation de position par perspective;
% 	\item rédaction d'un tutoriel sur la calibration de la stéréo-vision;
% 	\item rédaction d'un tutoriel sur la prévisualisation des flux ROS\@;
% 	\item rédaction de la partie ``Détection des plots'' dans ``Réalisation logiciel'';
% 	\item rédaction de la partie ``Estimation de la position des plots'' dans ``Réalisation logiciel''. \\
% \end{itemize}

% Temps estimé: 14h.


% %---------------- Christophe Cluizel -------------------------
% \subsection{Christophe Cluizel}

% Tâches effectuées:
% \begin{itemize}
% 	\item recherche documentaire sur la stéréo-vision;
% 	\item mise en place de l'architecture du rapport;
% 	\item intégration des informations de la semaine SOSI\@;
% 	\item rédaction de l'introduction;
% 	\item prise en main du logiciel GanttProject;
% 	\item création et intégration du planning et répartition des tâches;
% 	\item rédaction partie ``Problèmes rencontrés et solutions'';
% 	\item rédaction du compte rendu du 26/11;
% 	\item mise en page du rapport;
% 	\item relecture globale et correction orthographique. \\
% \end{itemize}

% Temps estimé: 11h.
