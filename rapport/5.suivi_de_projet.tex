\chapter{Suivi de projet}
Les comptes-rendus retracent les objectifs, les réalisations, ainsi que le temps estimé pour réaliser ces objectifs.

%============= Compte rendu à la date du 08/02 ==============
\section{Compte rendu à la date du 08/02}
L'objectif de ces premières semaines étaient de prendre en main le système Optitrack. Pour cela, plusieurs sous objectifs étaient à atteindre: ranger la ``salle drone'', installer les caméras, prendre en main le logiciel Motive, calibrer les caméras, détecter les marqueurs et obtenir leur position, étudier le SDK camera et le SDK NatNet.

%---------------- Alexandre Brehmer -------------------------
\subsection{Alexandre Brehmer}
Tâches effectuées:
\begin{itemize}
	\item \TODO{pouet} \\
\end{itemize}

Temps estimé:\TODO{????????h.}


%---------------- Christophe Cluizel -------------------------
\subsection{Christophe Cluizel}

Tâches effectuées:
\begin{itemize}
	\item Rangement de la salle et optimisation de l'espace;
	\item Installation du système Optitrack;
	\item Prise en main du logiciel Motive;
	\item Calibration des caméras (Motive);
	\item Détection des marqueurs (Motive);
	\item Calcul des positions des marqueurs dans l'espace 3D\@ (Motive);
	\item Étude du SDK camera et SDK NatNet. \\
\end{itemize}

Temps estimé: 19h.


%============= Compte rendu à la date du 20/02 ==============
\section{Compte rendu à la date du 20/02}
L'objectif principal de ces 2 semaines était de réussir à rassembler au sein d'un même code source les instructions de récupération de la position du drone provenant du serveur Motive et des instructions pour le déplacement du drone, le tout dans un environnement Windows avec le langage C++.

%---------------- Alexandre Brehmer -------------------------
\subsection{Alexandre Brehmer}
Tâches effectuées:
\begin{itemize}
	\item \TODO{pouet} \\
\end{itemize}

Temps estimé:\TODO{????????h}.


%---------------- Christophe Cluizel -------------------------
\subsection{Christophe Cluizel}

Tâches effectuées:
\begin{itemize}
	\item Installation et test de l'API de l'ARDrone;
	\item Recherche documentaire sur l'API du drone;
	\item Installation de différentes API du drone pour étudier le code source;
	\item Lecture de la documentation de l'API officielle de l'ARDrone;
	\item Installation et test d'un client Linux pour le serveur Motive;
	\item Programmation et premier test de récupération de la position du drone;
	\item Programmation et premier test de déplacement du drone. \\
\end{itemize}

Temps estimé: 13h.
