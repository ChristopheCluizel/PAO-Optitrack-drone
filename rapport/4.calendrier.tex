\chapter{Calendrier et répartition des tâches}

%============= Calendrier ===================
\section{Calendrier}

	Nous travaillons sur la base d'objectifs d'une durée de 2 semaines maximum. Cela permet de faire un point toutes les 2 semaines avec le responsable du PAO, M. Guerrero, afin d'orienter les objectifs suivants.

% ============= Répartition des tâches ===================
\section{Répartition des tâches}
	Les premières phases du PAO ont été réalisées en peer-programming, c'est-à-dire en binôme sur une même tache. Cela était indispensable pour la mise en place d'un environnement de développement viable. En effet, il a fallu rechercher et prendre en main les différents outils pour avoir une base stable et commune. \\

	La répartition des tâches s'est découpée comme présentée sur la figure suivante.


	% \begin{figure}
	% \hspace{-1cm}
	% \begin{tabular}{|@{}l@{}l}
	% 	\includegraphics[width=24cm,angle=90]{images/calendrier.png}
	% &
	% 	\includegraphics[width=24cm,angle=90]{images/repartitionCharge.png}
	%  \end{tabular}
	%  \caption{Rétro-calendrier et Répartition des tâches}
	%  \end{figure}
